\documentclass[10pt,twoside,openright,chapterprefix]{scrbook} %scrreprt, book, report,

\usepackage[papersize={170mm,225mm},
   margin=20mm,
   includehead,
   includefoot]{geometry}

\usepackage[%
	pagebackref=true,
	citecolor=blue,
	linkcolor=blue,
	colorlinks=true,
	urlcolor=cyan,
	pdfborder={0 0 0}
	]{hyperref}

% Use utf-8 encoding for foreign characters
\usepackage[utf8]{inputenc}
\usepackage[spanish,mexico]{babel}

\usepackage{soul}
\usepackage{amsthm, amssymb, amsmath}
\usepackage{subfigure}
\usepackage{algcompatible}
\usepackage{comment}
\usepackage{tikz}
\usepackage{graphicx}

\usepackage{mathtools}

\usepackage[numbers]{natbib}
% Completa en texto loss elementos de referencia
\usepackage[capitalize]{cleveref}
% Muestra los identificadores de ecuaciones y referencias
\usepackage[notref,notcite,color]{showkeys}

\usepackage{lipsum}


% Style theme  -----------------------------------------------------------------
\usepackage[Sonny]{fncychap}  %Sonny, Lenny, Glenn, Conny, Rejne and Bjarne
% Las siguientes lineas son necesarias pues las opciones de \sf y \bf ya no
% tienen soporte en KOMA.
\ChTitleVar{\bfseries\Large\rmfamily}
\ChNameVar{\bfseries\Large\sffamily}

%  Definición de bloques -------------------------------------------------------
\newtheorem{theorem}{Teorema}
\newtheorem{corollary}{Corolario}
\newtheorem{definition}{Definición}
\newtheorem{lemma}{Lema}
\newtheorem{exercise}{Exercise}
\newtheorem{remark}{Nota}
\newtheorem{example}{Example}
\newtheorem{warning}{Warning}
\newtheorem{proposition}[theorem]{Proposición}

% Definición de variables ------------------------------------------------------

\newcommand{\N}{\mathsf{N}}
\newcommand{\E}{\mathbb{E}}
\newcommand{\G}{\mathcal{G}}
\newcommand{\R}{\mathbb{R}}
\def\grad{ \mbox{grad}}
\def\curl{ \mbox{curl}}
\def\div{ \mbox{div}}
\def\U{\ensuremath {\cal U}}
\def\S{\ensuremath {\cal S}}
\def\V{\ensuremath {\cal V}}
\def\R{\ensuremath {\cal R}}
\def\tr{\ensuremath {\mbox{tr}}}

\graphicspath{{./imagenes/}}

% Título y autor ---------------------------------------------------------------
\title{Titulo del trabajo}
\author{Estudiante}
\date{}

% Esqueleto del documento ------------------------------------------------------
\begin{document}

% Primero lo portada
%!TEX root = ../maestria.tex

\thispagestyle{empty}
\setcounter{page}{1}
\begin{center}\vspace{70pt}
\begin{tabular}{c}
\hline
\Large \emph{\textsc{Instituto Tecnológico Autónomo de México}}\\
\hline
\end{tabular}\\
\vspace{20pt}
\includegraphics[width=.8\linewidth]{LOGO_ITAM.jpeg}\\
\vspace{20pt}
\huge Título del trabajo \\
\vspace{20pt}
\normalsize {\Large TESIS/TESINA/CASO}\\
\vspace{15pt}
QUE PARA OBTENER EL GRADO DE\\
\vspace{3pt}
{\Large Maestrx en Ciencia de Datos}\\
\vspace{27pt}
PRESENTA\\
\vspace{4pt}
{\Large Estudiante} \\
\vspace{23pt} % Para un caso se borra la siguiente línea
ASESORA\\
{\Large Alguien} \\
\vspace{23pt}
\vspace{1em}
\begin{tabular}{lcr}
MÉXICO, D.F. & \hspace{60pt} & 2021
\end{tabular}
\end{center}
%\newpage
%\setcounter{page}{1}
\pagenumbering{roman}
\mbox{ }
\vspace{70pt}
\mbox{ }

\section*{\phantom{Declaración}}
``Con fundamento en los artículos 21 y 27 de la Ley Federal de Derecho de Autor
y como titular de los derechos moral y patrimonial de la obra titulada
\textbf{``Título de trabajo''}, otorgo de manera gratuita y permanente al
Instituto Tecnológico Autónomo de México y a la biblioteca Raúl Billères Jr.,
autorización para que fijen la obra en cualquier medio, incluido el electrónico,
y la divulguen entre sus usuarios, profesores, estudiantes o terceras personas,
sin que pueda percibir por tal divulgación una contraprestación'' \vspace{2em}
\\ \begin{center} \vspace{2em}
    {Estudiante } \\
    \phantom{espacio} \vspace{5em}
    \underline{\phantom{\small{ALFREDO GARBUNO IÑIGO} } }\\
    \small{FECHA}      \\
    \phantom{espacio} \vspace{5em}
    \underline{\phantom{\small{ALFREDO GARBUNO IÑIGO} } }\\
    \small{FIRMA}
      \end{center}

%% Dedicatoria
\newpage
\thispagestyle{empty}
$\,$
\vspace{5.7cm}
\begin{flushright}
{\em {\Large Dedicatoria}}
\end{flushright}
\clearpage

\thispagestyle{empty}
\section*{Agradecimientos}

\noindent

\maketitle
\section*{Prólogo}

En esta tesina se estudian $\ldots$.
La estructura de este trabajo es el siguiente:

\begin{enumerate}
  \item {\bfseries Introducci\'on}.
  \item[] {\bfseries $\vdots$}
  \item[$n.$] {\bfseries Conclusiones}.
\end{enumerate}

\pagenumbering{arabic}
\tableofcontents

% Añade capítulos en orden
% !TEX root = ../maestria.tex
% La linea de arriba permite compile desde este archivo. 

\chapter{Introducción}

\lipsum
\chapter{Ejemplos}

Esto es un ejemplo de una cita \citep{Stuart2018}. 


% En emacs las lineas de abajo permiten compilar desde un archivo de capitulo
% ver: TeX-master-file-ask
%%% Local Variables:
%%% mode: latex
%%% TeX-master: "../maestria"
%%% End:

% \include{./capitulos/dist}
% \include{./capitulos/ipm}
% \include{./capitulos/smo}
% \include{./capitulos/conc}

% \appendix
%\include{./capitulos/appendix}

\nocite{*}
\bibliography{referencias}
\bibliographystyle{plainnat}


\end{document}

%%% Local Variables:
%%% mode: latex
%%% TeX-master: t
%%% End:
